\subsection*{Types and connectivity in artificial structures}
\begin{figure}
  \centering 
  \centerline{\includegraphics[width=4in]{mos6502.ai}}
  \caption{Discovering connectivity and type the 6502 microprocessor.
    a.) is the micrograph of the original microprocessor, with the
    region containing the registers under study highlighted. b.) Our
    graph consists of the interconnections of MOS field-effect
    transistors with three terminals, Gate, C1, and C2. Reconstruction
    technique did not permit resolution of C1 and C2 into source and
    drain. c.) The spatial distribution of the transistors in each
    cluster show a clear pattern d.) The clusters and connectivity
    versus distance for connections between Gate and C1, Gate and C2,
    and C1 and C2 terminals on a transistor. Red and teal types have a
    terminal pulled down to ground and mostly function as
    inverters. Purple class are clocked, stateful transistors, orange
    control the ALU and yellow control the special data bus (SDB).}
  \label{fig:mos6502}
\end{figure}


To show the applicability of our method to other connectome-style
datasets, we obtained the spatial location and interconnectivity of
the transistors in a classic microprocessor, the MOS Technology 6502
(used in the Apple II) \autocite{James2010}. Computer architects use
common patterns of transistors when designing circuits, with each
transistor having a ``type'' in the circuit. We identified a region of
the processor with complex but known structure containing the primary
8-bit registers X, Y, and S (fig~\ref{fig:mos6502}).

Our algorithm identifies areas of spatial homogeneity that mirror the
known structure in the underlying architectural circuit, segmenting
transistor types recognizable to computer architects. Using the
original schematics, we see that one identified type contains the
``clocked'' transistors, which retain digital state. Two other types
contain transistors with pins C1 or C2 connected to ground, mostly
serving as inverters.  An additional identified type controls the
behavior of the three registers of interest (X, Y, and S) with respect
to the SB data bus, either allowing them to latch or drive data from
the bus. The repeat patterns of spatial connectivity are visible in
figure~\ref{fig:mos6502}c, showing the man-made horizontal and
vertical layout of the same types of transistors.

